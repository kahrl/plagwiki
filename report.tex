\documentclass[ngerman,final,fontsize=12pt,paper=a4,twoside,bibliography=totoc,BCOR=8mm,draft=false]{scrartcl}

\usepackage[T1]{fontenc}
\usepackage{babel}
\usepackage[utf8]{inputenx}
\usepackage[sort&compress,square]{natbib}
\usepackage[babel]{csquotes}
\usepackage[hyphens]{url}
\usepackage[draft=false,final,plainpages=false,pdftex]{hyperref}
\usepackage{eso-pic}
\usepackage{fixltx2e}
\usepackage{graphicx}
\usepackage{xcolor}
\usepackage{pdflscape}
\usepackage{colortbl}
\usepackage{longtable}
\usepackage{multirow}
\usepackage{framed}
\usepackage{textcomp}
\usepackage{scrtime}

\usepackage[charter,sfscaled]{mathdesign}

%\usepackage[spacing=true,tracking=true,kerning=true,babel]{microtype}
\usepackage[spacing=true,kerning=true,babel]{microtype}

\author{VroniPlag}

\title{Bericht 20110618}
\subtitle{Gemeinschaftliche Dokumentation von Plagiaten in der Dissertation „Amerika: das Experiment des Fortschritts. Ein Vergleich des politischen Denkens in Europa und in den USA“ von Prof.~Dr.~Margarita Mathiopoulos}
\publishers{\normalsize\url{http://de.vroniplag.wikia.com/wiki/Mm}}

\hypersetup{%
        pdfauthor={VroniPlag},%
        pdftitle={Bericht 20110618 --- Gemeinschaftliche Dokumentation von Plagiaten in der Dissertation „Amerika: das Experiment des Fortschritts. Ein Vergleich des politischen Denkens in Europa und in den USA“ von Prof.~Dr.~Margarita Mathiopoulos},%
        pdflang={en},%
        %pdfduplex={DuplexFlipLongEdge},%
        %pdfprintscaling={None},%
        %linktoc=all,%
        colorlinks,%
        linkcolor=black,%
        citecolor=green!50!black,%
        filecolor=blue,%
        urlcolor=blue,%
        linkbordercolor={1 0 0},%
        citebordercolor={0 0.5 0},%
        filebordercolor={0 0 1},%
        urlbordercolor={0 0 1},%
}

\definecolor{shadecolor}{rgb}{0.95,0.95,0.95}

\newenvironment{fragment}
        {\begin{snugshade}}
        {\end{snugshade}
                \penalty-200
                \vskip 0pt plus 10mm minus 5mm}
\newenvironment{fragmentpart}[1]
        {\indent\textbf{#1}\par\penalty500\noindent}
        {\par}
\newcommand{\BackgroundPic}
        {\put(0,0){\parbox[b][\paperheight]{\paperwidth}{%
                \vfill%
                \centering%
                \includegraphics[width=\paperwidth,height=\paperheight,%
                        keepaspectratio]{background.png}%
                \vfill%
        }}}
\newcommand{\hrulesep}{%
        \nointerlineskip\vspace{\baselineskip}%
        \hrule\par%
        \nointerlineskip\vspace{\baselineskip}%
}


\setkomafont{section}{\large}
\addtokomafont{disposition}{\normalfont\boldmath\bfseries}
\urlstyle{rm}

\date{\today, \thistime}
%\date{19. April 2011, 17:00}

\begin{document}

%\AddToShipoutPicture*{\BackgroundPic}
\maketitle\thispagestyle{empty}
%\ClearShipoutPicture

\tableofcontents

% Encountered a pre tag
\begin{verbatim}
                                       ENTWURF
\end{verbatim}
 % Ignoring infobox table
 % Ignoring toc table
% Ignoring script
 % Encountered a h2 tag
\section{% Ignoring editsection span
 % Encountered a span tag
%   Attributes: {u'class': u'mw-headline', u'id': u'Zusammenfassung.2F.C3.9Cberblick'}
Zusammenfassung/Überblick}
 % Encountered a p tag

Gegenstand dieses Berichts ist die Untersuchung der Dissertation % Encountered a i tag
\textit{Amerika: Das Experiment des Fortschritts --- Ein Vergleich des politischen Denkens in den USA und Europa} von Margarita Mathiopoulos auf Plagiatstellen. 

% Encountered a p tag

Die von dem Politologen Karl Dietrich Bracher betreute und 1986 an der Universität Bonn eingereichte Arbeit ist bereits seit 1989 einem Plagiatsverdacht ausgesetzt. Die gründliche Quellenanalyse auf VroniPlag Wiki weist nun Plagiatstellen im Einzelnen nach. 

% Encountered a p tag

Bisher wurden über 330 Plagiatstellen auf über 130 von 350 Seiten der Dissertation gefunden. 77\% dieser Übernahmen sind vollkommen ohne Zitierung, bei 22\% der Übernahmen liegt eine irreführende Zitierung vor. 

% Encountered a p tag

Unter zahlreichen neu gefundenen Plagiatsquellen ist auch die Dissertation des damaligen Ehemanns % Encountered a a tag
%   Attributes: {u'href': u'/wiki/Kategorie:Pfl%C3%BCger_1983', u'title': u'Kategorie:Pfl\xfcger 1983'}
\href{http://de.vroniplag.wikia.com/wiki/Kategorie:Pfl\%C3\%BCger_1983}{Friedbert Pflüger}, der vier Jahre zuvor bei demselben Doktorvater an der Universität Bonn promoviert worden war (% Encountered a a tag
%   Attributes: {u'href': u'/wiki/Mm/Mathiopoulos-1987/278', u'title': u'Mm/Mathiopoulos-1987/278'}
\href{http://de.vroniplag.wikia.com/wiki/Mm/Mathiopoulos-1987/278}{S. 278-288}). Unbelegte und irreführend zitierte Passagen stammen sowohl aus Werken des Doktorvaters Karl Dietrich Bracher als auch von % Encountered a a tag
%   Attributes: {u'href': u'/wiki/Kategorie:Craig_1984', u'title': u'Kategorie:Craig 1984'}
\href{http://de.vroniplag.wikia.com/wiki/Kategorie:Craig_1984}{Gordon A. Craig}, dem Verfasser des Vorworts zur amerikanischen Ausgabe der Dissertation. 

 % Encountered a h2 tag
\section{% Ignoring editsection span
 % Encountered a span tag
%   Attributes: {u'class': u'mw-headline', u'id': u'Einleitung'}
Einleitung}
 % Encountered a p tag

Gegenstand dieses Berichts ist die Untersuchung der im Schöningh-Verlag 1987 veröffentlichten Dissertation % Encountered a i tag
\textit{Amerika: Das Experiment des Fortschritts --- Ein Vergleich des politischen Denkens in den USA und Europa} von Margarita Mathiopoulos auf Plagiatstellen. 

% Encountered a p tag

Geleitet wird die in diesem Bericht dokumentierte Untersuchung von Prinzipien wissenschaftlichen Arbeitens, die unter anderem in „% Encountered a a tag
%   Attributes: {u'href': u'http://de.vroniplag.wikia.com/wiki/Wissenschaft_-_organisierter_Skeptizismus', u'class': u'text'}
\href{http://de.vroniplag.wikia.com/wiki/Wissenschaft_-_organisierter_Skeptizismus}{Wissenschaft --- organisierter Skeptizismus}“ dargelegt sind. Forschung und Wissenschaft sind kooperative Prozesse, in denen die Forschungsergebnisse, Texte und Ideen anderer öffentlich, transparent und --- vor allem in der Urheberschaft --- nachvollziehbar weiterentwickelt werden. 

% Encountered a p tag

Die bislang dokumentierten Fragmente erlauben es der akademischen und allgemeinen Öffentlichkeit, sich ein eigenes Bild des Falls zu machen. Eine detaillierte, kontinuierlich erweiterte Dokumentation der Projektergebnisse ist unter % Encountered a a tag
%   Attributes: {u'href': u'http://de.vroniplag.wikia.com/wiki/Mm', u'class': u'free'}
\url{http://de.vroniplag.wikia.com/wiki/Mm} zu finden. 

 % Encountered a h2 tag
\section{% Ignoring editsection span
 % Encountered a span tag
%   Attributes: {u'class': u'mw-headline', u'id': u'Zur_Rezeptionsgeschichte_der_Arbeit'}
Zur Rezeptionsgeschichte der Arbeit}
 % Encountered a p tag

Die Dissertation --- nach Aussagen der Autorin entstand sie in den Jahren 1981 bis 1986 --- und ein damit verbundener Forschungsaufenthalt an der Universität Harvard wurde in den Jahren 1981 bis 1983 durch die Friedrich-Naumann-Stiftung mit Mitteln des damaligen Bundesministeriums für Wissenschaft und Forschung gefördert. Die Arbeit mit dem Thema „Geschichte und Fortschritt im Denken Amerikas: Ein europäisch-amerikanischer Vergleich“ wurde 1986 an der Philosophischen Fakultät der Rheinischen Friedrich-Wilhelms-Universität Bonn eingereicht und im November 1986 mit „magna cum laude“ benotet (vgl. % Encountered a a tag
%   Attributes: {u'href': u'http://www.spiegel.de/spiegel/print/d-13522998.html', u'class': u'external text', u'rel': u'nofollow'}
\href{http://www.spiegel.de/spiegel/print/d-13522998.html}{Spiegel 13/1987}). Erstgutachter war der Politologe Prof. Dr. Karl Dietrich Bracher. 

% Encountered a p tag

1987 wurde sie im Schöningh-Verlag, Paderborn, mit einem Geleitwort Brachers veröffentlicht. 1988 erschien eine amerikanische Ausgabe bei Praeger, New York, mit einem Vorwort von Gordon A. Craig. 

% Encountered a p tag

Zu dieser Zeit gab es noch nicht die Möglichkeiten des Internet. Die Arbeit wurde also „klassisch“ unter Verwendung gedruckter Quellen (Monographien, Aufsätze in Zeitschriften etc.) verfasst. Primärquellen wurden nicht benutzt, gelegentlich verweist die Autorin auf ausführliche Gespräche mit Fachwissenschaftlern und Zeitzeugen. Wie der Verfasserin keine elektronischen Hilfsmittel zur Verfügung standen, war damals auch die programmunterstützte und vernetzte Plagiatsuche nicht möglich. 

% Encountered a p tag

Gleichwohl berichtete das Nachrichtenmagazin Der Spiegel bereits 1989 % Encountered a a tag
%   Attributes: {u'href': u'http://www.spiegel.de/spiegel/print/d-13496867.html', u'class': u'external text', u'rel': u'nofollow'}
\href{http://www.spiegel.de/spiegel/print/d-13496867.html}{(Heft 37/1989)}, zwei Jahre nach der Veröffentlichung, von „erstaunlichen Parallelen“ der Dissertation mit Büchern der Historiker Horst Dippel und Hans R. Guggisberg sowie des Politikwissenschaftlers Horst Mewes und hielt fest: „Sie schrieb an etlichen Stellen aus deren Arbeiten beinahe wörtlich ab, ohne die Passagen, wie es sich bei einer Dissertation ziemt, als Zitate auszuweisen“. Als Beispiele präsentierte der Spiegel drei übereinstimmende Abschnitte auf den Seiten 135-136, 217 und 260 von „Amerika“. 

% Encountered a p tag

Bracher hielt dazu fest, „daß an einigen Stellen die angewandten Arbeitsmethoden nicht wissenschaftlichen Gepflogenheiten entsprechen“, aber dadurch „der Kern der geistigen Leistung von Frau Mathiopoulos nicht beeinträchtigt“ sei (zitiert nach Spiegel 37/1989). 

% Encountered a p tag

Zwei Jahre später erschien im Band 36 (1991) der Amerikastudien eine Rezension der Dissertation von Kurt L. Shell (S. 567-568), in der auch auf den Plagiatsverdacht Bezug genommen wird. Dieser wird aber gegenüber der inhaltlichen „grundlegenden Schwächen“ als „nebensächlich“ bezeichnet. In derselben Ausgabe findet sich eine nicht namentlich gezeichnete Dokumentation mit dem Titel % Encountered a i tag
\textit{Die Rezeption deutscher Amerikanisten durch Margarita Mathiopoulos} (S. 545-559), in welcher die Liste der im Spiegel bereits als mögliche Quellen für Übernahmen genannten Arbeiten um ein weiteres Fachbuch (Schröder 1982), einen Fachartikel (Angermann 1979), eine politikwissenschaftliche Dissertation aus dem Jahre 1967 (Krakau 1967) und einen literaturwissenschaftlichen Aufsatz aus der Beilage % Encountered a i tag
\textit{Aus Politik und Zeitgeschichte} (Levine 1984) ergänzt wird. In der Dokumentation werden Textpassagen aus den Quellen den entsprechenden aus der Dissertation explizit gegenübergestellt und summarisch insgesamt 33 Seiten angeführt, die Parallelen aufweisen. 

% Encountered a p tag

Diese Tatsachen sind der Universität Bonn bekannt: Der Spiegel zitiert ein inzwischen erstelltes Gutachten der Philosophischen Fakultät der Universität Bonn (% Encountered a a tag
%   Attributes: {u'href': u'http://www.spiegel.de/spiegel/print/d-23905874.html', u'class': u'external text', u'rel': u'nofollow'}
\href{http://www.spiegel.de/spiegel/print/d-23905874.html}{Spiegel 34/2002}) damit, „dass in der Arbeit 'in wörtlicher und sinngemäßer Wiedergabe mehr übernommen' sei, 'als es die Zitatnachweise' erkennen lassen.“ 

% Encountered a p tag

Laut Frankfurter Allgemeiner Sonntagszeitung vom 20. Februar 2011 stellte Mathiopoulos bzgl. ihrer Dissertation fest: 

 % Encountered a blockquote tag
\begin{quote}
„Nachdem seinerzeit eine Kommission der Universität Bonn, wo ich promoviert habe, die Vorwürfe untersucht hatte, die TU Braunschweig aufgrund des Befundes der Universität Bonn und vier wissenschaftlichen Gutachten von renommierten Kollegen, die meine wissenschaftlichen Arbeiten, Bücher und Aufsätze begutachteten, mich 1995 zur ersten Honorarprofessorin in ihrer 250-jährigen Geschichte beriefen, und ich 2002 von der Historischen Fakultät der Universität Potsdam zur Honorarprofessorin ernannt wurde --- aufgrund von drei weiteren wissenschaftlichen Gutachten von renommierten Kollegen, ist das Thema abschließend geklärt.“\end{quote}
 % Encountered a p tag

Wiederholt reagierte sie in der Vergangenheit auf in den Medien geäußerte Zweifel an der Integrität ihrer Dissertation mit Gegendarstellungen. 2002 erwirkte Margarita Mathiopoulos eine einstweilige Verfügung auf Unterlassung gegen die Berliner Zeitung und verlangte „von der Zeitung eine prominent aufgemachte Gegendarstellung auf der Titelseite.“ (% Encountered a a tag
%   Attributes: {u'href': u'http://www.spiegel.de/spiegel/print/d-23905874.html', u'class': u'external text', u'rel': u'nofollow'}
\href{http://www.spiegel.de/spiegel/print/d-23905874.html}{Der Spiegel 34/2002}) Diesen Antrag zog sie in der Hauptverhandlung wieder zurück. 

% Encountered a p tag

Die festgestellten Mängel haben jedoch bis heute zu keinerlei Konsequenzen geführt. 

 % Encountered a h2 tag
\section{% Ignoring editsection span
 % Encountered a span tag
%   Attributes: {u'class': u'mw-headline', u'id': u'Vorgehensweise_auf_VroniPlag_Wiki'}
Vorgehensweise auf VroniPlag Wiki}
 % Encountered a p tag

Auf der Plattform Vroniplag Wiki dokumentieren Benutzer Plagiate in Dissertationen. Wie auch in der Wikipedia ist die Mitarbeit grundsätzlich für jeden Internetnutzer auch ohne Anmeldung offen. 

% Encountered a p tag

Als Plagiat identifizierte Textpassagen werden in sogenannten % Encountered a a tag
%   Attributes: {u'href': u'/wiki/Skm/Fragmente', u'title': u'Skm/Fragmente'}
\href{http://de.vroniplag.wikia.com/wiki/Skm/Fragmente}{Fragmenten} erfasst und nach Plagiatsarten kategorisiert (s. Anhang, „Definition von Plagiatskategorien“). In den Fragmenten steht der Textabschnitt der Dissertation dem nachweislich früher erschienenen Originaltext zitatweise gegenüber, um einen direkten Vergleich der Passagen möglich zu machen. Während und neben der Erfassung der Fragmente erfolgt eine Plausibilitätsprüfung im Rahmen einer Zweitsichtung durch einen anderen Wiki-Benutzer nach dem „Vier-Augen-Prinzip“. 

% Encountered a p tag

Bei der Untersuchung der vorliegenden Arbeit wurden in einem ersten Schritt im April 2011 die seit 1991 aus den Amerikastudien bekannten Textübernahmen in solchen Fragmenten dokumentiert und statistisch erfasst. Ab Mai 2011 wurden im zweiten Schritt vor allem die in den Amerikastudien genannten Quellen komplett gesichtet und mit der Dissertation abgeglichen. Im dritten Abschnitt erfolgt der Abgleich gegen andere in der Arbeit verwendete und angeführte Quellen und die Suche nach nicht im Literaturverzeichnis angegebenen Vorlagen. Dieser letzte Abschnitt der Untersuchung ist noch nicht abgeschlossen. 

 % Encountered a h2 tag
\section{% Ignoring editsection span
 % Encountered a span tag
%   Attributes: {u'class': u'mw-headline', u'id': u'Vorl.C3.A4ufige_Ergebnisse'}
Vorläufige Ergebnisse}
 % Encountered a p tag

Bisher wurden in der Dissertation mehr als 330 Textübernahmen auf über 130 Seiten aus Werken anderer Autoren gefunden, die als Plagiate einzustufen sind. Diese Fundstellen sind im Anhang A (Textnachweise) im Einzelnen dokumentiert. 

 % Encountered a h3 tag
\subsection{% Ignoring editsection span
 % Encountered a span tag
%   Attributes: {u'class': u'mw-headline', u'id': u'Aufschl.C3.BCsselung_der_Fundstellen'}
 Aufschlüsselung der Fundstellen }
 % Encountered a p tag

Bei der überwiegenden Mehrheit der Plagiatsfundstellen handelt es sich um unzitierte verschleierte oder komplett kopierte Passagen aus anderen Werken (77\%). Die nicht sachgemäß zitierten Übernahmen aus anderen Quellen stehen quantitativ zurück (22\%). 

% Encountered a p tag

Die große Anzahl der durch aktive Redigierung und Synonymersetzung verschleierten Textübernahmen ohne jeden Quellhinweis schließt ein nachlässiges Arbeiten aus und setzt ein bewusstes Handeln der Autorin zwingend voraus. Die unsachgemäß zitierten --- aber immerhin mit einem Quellverweis versehenen --- Übernahmen treten in der Anzahl der Funde zwar hinter der ersten Gruppe zurück, jedoch nicht in ihrer Bedeutung. „Bauernopfer“-Zitate und andere Falschzitierungen haben eine hohe verschleiernde Wirkung und sind zur Vortäuschung einer selbständigen wissenschaftlichen Leistung besonders geeignet. Der besonderen Verwendung der „Bauernopfer“-Zitierung im vorliegenden Werk widmet sich der Abschnitt % Encountered a a tag
%   Attributes: {u'href': u'/wiki/Mm/Bericht-Entwurf#Verschleierung_von_Text_und_Quellenangaben', u'title': u'Mm/Bericht-Entwurf'}
\href{http://de.vroniplag.wikia.com/wiki/Mm/Bericht-Entwurf\#Verschleierung_von_Text_und_Quellenangaben}{„Verschleierung von Text und Quellenangaben“}. 

% Encountered a p tag

% Encountered a br tag
\ifhmode\\\fi
 

 % Encountered a table tag
%   Attributes: {u'class': u'wikitable'}
% Encountered a caption tag
% Encountered a tr tag
% Encountered a th tag
%   Attributes: {u'style': u'width:75%;'}
% Encountered a b tag
% Encountered a th tag
% Encountered a b tag
% Encountered a tr tag
% Encountered a th tag
%   Attributes: {u'colspan': u'2', u'style': u'background-color:#ffdead;'}
% Encountered a i tag
% Encountered a tr tag
%   Attributes: {u'style': u'text-align:right;vertical-align:top;'}
% Encountered a td tag
%   Attributes: {u'style': u'text-align:left'}
% Encountered a p tag
% Encountered a a tag
%   Attributes: {u'href': u'/wiki/Kategorie:Verschleierung', u'title': u'Kategorie:Verschleierung'}
% Encountered a b tag
% Encountered a br tag
% Encountered a td tag
%   Attributes: {u'style': u'font-weight:bold;padding:3 3;'}
% Encountered a tr tag
%   Attributes: {u'style': u'text-align:right;vertical-align:top;'}
% Encountered a td tag
%   Attributes: {u'style': u'text-align:left'}
% Encountered a p tag
% Encountered a a tag
%   Attributes: {u'href': u'/wiki/Kategorie:KomplettPlagiat', u'title': u'Kategorie:KomplettPlagiat'}
% Encountered a b tag
% Encountered a br tag
% Encountered a td tag
%   Attributes: {u'style': u'font-weight:bold;padding:3 3;'}
% Encountered a tr tag
%   Attributes: {u'style': u'text-align:right;vertical-align:top;'}
% Encountered a th tag
%   Attributes: {u'colspan': u'2', u'style': u'background-color:#ffdead;'}
% Encountered a i tag
% Encountered a tr tag
%   Attributes: {u'style': u'text-align:right;vertical-align:top;'}
% Encountered a td tag
%   Attributes: {u'style': u'text-align:left'}
% Encountered a p tag
% Encountered a a tag
%   Attributes: {u'href': u'/wiki/Kategorie:BauernOpfer', u'title': u'Kategorie:BauernOpfer'}
% Encountered a b tag
% Encountered a br tag
% Encountered a td tag
%   Attributes: {u'style': u'font-weight:bold;padding:3 3;'}
% Encountered a tr tag
%   Attributes: {u'style': u'text-align:right;vertical-align:top;'}
% Encountered a td tag
%   Attributes: {u'style': u'text-align:left'}
% Encountered a a tag
%   Attributes: {u'href': u'/wiki/Kategorie:Versch%C3%A4rftesBauernopfer', u'title': u'Kategorie:Versch\xe4rftesBauernopfer'}
% Encountered a b tag
% Encountered a br tag
% Encountered a td tag
%   Attributes: {u'style': u'font-weight:bold;padding:3 3;'}
% Encountered a tr tag
% Encountered a td tag
%   Attributes: {u'colspan': u'2', u'style': u'text-align:center'}
% Encountered a small tag
\begin{table}[htp]
\centering
\caption{Anzahl Fundstellen nach Plagiatskategorie }
\begin{longtable}{|p{9.75cm}|p{3.25cm}|p{0cm}}
\hline
\cellcolor[rgb]{0.949,0.949,0.949}\parbox[t]{9.75cm}{\textbf{\textbf{Kategorie} }} &\cellcolor[rgb]{0.949,0.949,0.949}\parbox[t]{3.25cm}{\textbf{\textbf{Anzahl} }} &\\
\multicolumn{2}{p{13.0cm}}{\cellcolor[rgb]{1.000,0.871,0.678}\parbox[t]{13.0cm}{\textbf{\textit{unzitiert} }}} &\\
\parbox[t]{9.75cm}{ \href{http://de.vroniplag.wikia.com/wiki/Kategorie:Verschleierung}{\textbf{Verschleierungen}}:\newline
umformulierte Texte, die weder als Paraphrase noch als Zitat kenntlich gemacht wurden.  } &\parbox[t]{3.25cm}{286 } &\\
\parbox[t]{9.75cm}{ \href{http://de.vroniplag.wikia.com/wiki/Kategorie:KomplettPlagiat}{\textbf{Komplettplagiate}}:\newline
Komplette Abschnitte der Quelle wurden wörtlich oder fast wörtlich und ohne Zitat übernommen.  } &\parbox[t]{3.25cm}{10 } &\\
\multicolumn{2}{p{13.0cm}}{\cellcolor[rgb]{1.000,0.871,0.678}\parbox[t]{13.0cm}{\textbf{\textit{mit Zitat} }}} &\\
\parbox[t]{9.75cm}{ \href{http://de.vroniplag.wikia.com/wiki/Kategorie:BauernOpfer}{\textbf{Bauernopfer}}:\newline
ein Verweis zur Quelle ist für einen kleinen Teil der Übernahme angebracht, während größere Abschnitte in unmittelbarer Umgebung ohne Zitatnachweis übernommen werden.  } &\parbox[t]{3.25cm}{75 } &\\
\parbox[t]{9.75cm}{\href{http://de.vroniplag.wikia.com/wiki/Kategorie:Versch\%C3\%A4rftesBauernopfer}{\textbf{Verschärfte Bauernopfer}}:\newline
wie Bauernopfer, jedoch enthält der Fußnotentext verschleiernde Wendungen wie „vgl. ...“, „so auch ...“ mit Bezug auf den Originaltext. } &\parbox[t]{3.25cm}{9 } &\\
\multicolumn{2}{p{13.0cm}}{\parbox[t]{13.0cm}{ {\small (Stand der Tabelle: Sun, 26 Jun 2011 08:40:24 +0000)} }} &\\
\hline
\end{longtable}
\end{table}
 % Encountered a h3 tag
\subsection{% Ignoring editsection span
 % Encountered a span tag
%   Attributes: {u'class': u'mw-headline', u'id': u'Vergleich_der_Plagiatsuntersuchungen'}
Vergleich der Plagiatsuntersuchungen}
 % Encountered a p tag

Bereits in den Amerikastudien 1991 wurden in 33 Seiten der Dissertation Übernahmen aus 7 verschiedenen Quellen gefunden und zum Teil in einer Synopse den Originalstellen gegenübergestellt. Die dort angeführten Funde wurden im VroniPlag Wiki neu in 62 Fundstellen erfasst und die bisher nur stichprobenhaft untersuchten Quellen systematisch durchforstet. VroniPlag-Mitarbeiter konnten nochmals mehr als 200 weitere Übernahmen aus den bekannten Quellen dokumentieren. 

 % Encountered a table tag
%   Attributes: {u'class': u'wikitable'}
% Encountered a caption tag
% Encountered a tr tag
% Encountered a th tag
%   Attributes: {u'style': u'background-color:#ffdead; width:40%', u'rowspan': u'2'}
% Encountered a th tag
%   Attributes: {u'colspan': u'2', u'style': u'background-color:#ffdead;'}
% Encountered a th tag
%   Attributes: {u'colspan': u'2', u'style': u'background-color:#ffdead;'}
% Encountered a tr tag
% Encountered a th tag
%   Attributes: {u'style': u'background-color:#ffdead;'}
% Encountered a a tag
%   Attributes: {u'href': u'http://books.google.com/books?hl=de&id=wGwrAQAAIAAJ&q=mathiopoulos#search_anchor', u'class': u'external text', u'rel': u'nofollow'}
% Encountered a br tag
% Encountered a br tag
% Encountered a th tag
%   Attributes: {u'style': u'background-color:#ffdead;'}
% Encountered a br tag
% Encountered a th tag
%   Attributes: {u'style': u'background-color:#ffdead;'}
% Encountered a a tag
%   Attributes: {u'href': u'http://books.google.com/books?hl=de&id=wGwrAQAAIAAJ&q=mathiopoulos#search_anchor', u'class': u'external text', u'rel': u'nofollow'}
% Encountered a br tag
% Encountered a br tag
% Encountered a th tag
%   Attributes: {u'style': u'background-color:#ffdead;'}
% Encountered a br tag
% Encountered a tr tag
%   Attributes: {u'style': u'text-align:right'}
% Encountered a td tag
%   Attributes: {u'style': u'text-align:left;'}
% Encountered a a tag
%   Attributes: {u'href': u'/wiki/Kategorie:Angermann_1979', u'title': u'Kategorie:Angermann 1979'}
% Encountered a td tag
% Encountered a td tag
% Encountered a td tag
% Encountered a td tag
% Encountered a tr tag
%   Attributes: {u'style': u'text-align:right'}
% Encountered a td tag
%   Attributes: {u'style': u'text-align:left'}
% Encountered a a tag
%   Attributes: {u'href': u'/wiki/Kategorie:Dippel_1985', u'title': u'Kategorie:Dippel 1985'}
% Encountered a td tag
% Encountered a td tag
% Encountered a td tag
% Encountered a td tag
% Encountered a tr tag
%   Attributes: {u'style': u'text-align:right'}
% Encountered a td tag
%   Attributes: {u'style': u'text-align:left'}
% Encountered a a tag
%   Attributes: {u'href': u'/wiki/Kategorie:Guggisberg_1979a', u'title': u'Kategorie:Guggisberg 1979a'}
% Encountered a td tag
% Encountered a td tag
% Encountered a td tag
% Encountered a td tag
% Encountered a tr tag
%   Attributes: {u'style': u'text-align:right'}
% Encountered a td tag
%   Attributes: {u'style': u'text-align:left'}
% Encountered a a tag
%   Attributes: {u'href': u'/wiki/Kategorie:Krakau_1967', u'title': u'Kategorie:Krakau 1967'}
% Encountered a td tag
% Encountered a td tag
% Encountered a td tag
% Encountered a td tag
% Encountered a tr tag
%   Attributes: {u'style': u'text-align:right'}
% Encountered a td tag
%   Attributes: {u'style': u'text-align:left'}
% Encountered a a tag
%   Attributes: {u'href': u'/wiki/Kategorie:Levine_1984', u'title': u'Kategorie:Levine 1984'}
% Encountered a td tag
% Encountered a td tag
% Encountered a td tag
% Encountered a td tag
% Encountered a tr tag
%   Attributes: {u'style': u'text-align:right'}
% Encountered a td tag
%   Attributes: {u'style': u'text-align:left'}
% Encountered a a tag
%   Attributes: {u'href': u'/wiki/Kategorie:Mewes_1986', u'title': u'Kategorie:Mewes 1986'}
% Encountered a td tag
% Encountered a td tag
% Encountered a td tag
% Encountered a td tag
% Encountered a tr tag
%   Attributes: {u'style': u'text-align:right'}
% Encountered a td tag
%   Attributes: {u'style': u'text-align:left'}
% Encountered a a tag
%   Attributes: {u'href': u'/wiki/Kategorie:Schr%C3%B6der_1982', u'title': u'Kategorie:Schr\xf6der 1982'}
% Encountered a td tag
% Encountered a td tag
% Encountered a td tag
% Encountered a td tag
% Encountered a tr tag
%   Attributes: {u'style': u'font-weight:bold;background-color:#ffdead;text-align:right'}
% Encountered a td tag
%   Attributes: {u'style': u'text-align:left'}
% Encountered a td tag
% Encountered a td tag
% Encountered a td tag
% Encountered a td tag
% Encountered a tr tag
%   Attributes: {u'style': u'text-align:right'}
% Encountered a td tag
%   Attributes: {u'style': u'text-align:left'}
% Encountered a td tag
% Encountered a td tag
% Encountered a td tag
% Encountered a td tag
% Encountered a tr tag
%   Attributes: {u'style': u'font-weight:bold;background-color:#ffdead;text-align:right'}
% Encountered a td tag
%   Attributes: {u'style': u'text-align:left'}
% Encountered a td tag
% Encountered a td tag
% Encountered a td tag
% Encountered a td tag
% Encountered a tr tag
% Encountered a td tag
%   Attributes: {u'colspan': u'5', u'style': u'text-align:center'}
% Encountered a small tag
\begin{table}[htp]
\centering
\caption{Vergleich der Plagiatsuntersuchungen 1991 und 2011 }
\begin{longtable}{|p{5.2cm}|p{1.95cm}|p{1.95cm}|p{1.95cm}|p{1.95cm}|p{0cm}}
\hline
\cellcolor[rgb]{1.000,0.871,0.678}\parbox[t]{5.2cm}{~} &\multicolumn{2}{p{3.9cm}}{\cellcolor[rgb]{1.000,0.871,0.678}\parbox[t]{3.9cm}{\textbf{ Anzahl Fragmente }}} &\multicolumn{2}{p{3.9cm}}{\cellcolor[rgb]{1.000,0.871,0.678}\parbox[t]{3.9cm}{\textbf{ auf Anzahl Seiten }}} &\\
\multirow{-2}{*}{\cellcolor[rgb]{1.000,0.871,0.678}\parbox[t]{5.2cm}{\textbf{ Quelle }}} &\cellcolor[rgb]{1.000,0.871,0.678}\parbox[t]{1.95cm}{\textbf{ \href{http://books.google.com/books?hl=de\&id=wGwrAQAAIAAJ\&q=mathiopoulos\#search_anchor}{Amerika-\newline
studien}\newline
1991 }} &\cellcolor[rgb]{1.000,0.871,0.678}\parbox[t]{1.95cm}{\textbf{ VroniPlag\newline
2011 }} &\cellcolor[rgb]{1.000,0.871,0.678}\parbox[t]{1.95cm}{\textbf{ \href{http://books.google.com/books?hl=de\&id=wGwrAQAAIAAJ\&q=mathiopoulos\#search_anchor}{Amerika-\newline
studien}\newline
1991 }} &\cellcolor[rgb]{1.000,0.871,0.678}\parbox[t]{1.95cm}{\textbf{ VroniPlag\newline
2011 }} &\\
\parbox[t]{5.2cm}{ \href{http://de.vroniplag.wikia.com/wiki/Kategorie:Angermann_1979}{Angermann 1979} } &\parbox[t]{1.95cm}{ 10 } &\parbox[t]{1.95cm}{ 57 } &\parbox[t]{1.95cm}{ 6 } &\parbox[t]{1.95cm}{ 24 } &\\
\parbox[t]{5.2cm}{ \href{http://de.vroniplag.wikia.com/wiki/Kategorie:Dippel_1985}{Dippel 1985} } &\parbox[t]{1.95cm}{ 14 } &\parbox[t]{1.95cm}{ 30 } &\parbox[t]{1.95cm}{ 6 } &\parbox[t]{1.95cm}{ 9 } &\\
\parbox[t]{5.2cm}{ \href{http://de.vroniplag.wikia.com/wiki/Kategorie:Guggisberg_1979a}{Guggisberg 1979a} } &\parbox[t]{1.95cm}{ 7 } &\parbox[t]{1.95cm}{ 37 } &\parbox[t]{1.95cm}{ 4 } &\parbox[t]{1.95cm}{ 18 } &\\
\parbox[t]{5.2cm}{ \href{http://de.vroniplag.wikia.com/wiki/Kategorie:Krakau_1967}{Krakau 1967} } &\parbox[t]{1.95cm}{ 5 } &\parbox[t]{1.95cm}{ 37 } &\parbox[t]{1.95cm}{ 2 } &\parbox[t]{1.95cm}{ 19 } &\\
\parbox[t]{5.2cm}{ \href{http://de.vroniplag.wikia.com/wiki/Kategorie:Levine_1984}{Levine 1984} } &\parbox[t]{1.95cm}{ 10 } &\parbox[t]{1.95cm}{ 10 } &\parbox[t]{1.95cm}{ 5 } &\parbox[t]{1.95cm}{ 5 } &\\
\parbox[t]{5.2cm}{ \href{http://de.vroniplag.wikia.com/wiki/Kategorie:Mewes_1986}{Mewes 1986} } &\parbox[t]{1.95cm}{ 11 } &\parbox[t]{1.95cm}{ 55 } &\parbox[t]{1.95cm}{ 8 } &\parbox[t]{1.95cm}{ 29 } &\\
\parbox[t]{5.2cm}{ \href{http://de.vroniplag.wikia.com/wiki/Kategorie:Schr\%C3\%B6der_1982}{Schröder 1982} } &\parbox[t]{1.95cm}{ 5 } &\parbox[t]{1.95cm}{ 59 } &\parbox[t]{1.95cm}{ 2 } &\parbox[t]{1.95cm}{ 14 } &\\
\cellcolor[rgb]{1.000,0.871,0.678}\parbox[t]{5.2cm}{ bekannte Quellen } &\cellcolor[rgb]{1.000,0.871,0.678}\parbox[t]{1.95cm}{ 62 } &\cellcolor[rgb]{1.000,0.871,0.678}\parbox[t]{1.95cm}{ 285 } &\cellcolor[rgb]{1.000,0.871,0.678}\parbox[t]{1.95cm}{ 33 } &\cellcolor[rgb]{1.000,0.871,0.678}\parbox[t]{1.95cm}{ 105 } &\\
\parbox[t]{5.2cm}{ aus neuen Quellen } &\parbox[t]{1.95cm}{ } &\parbox[t]{1.95cm}{ 96 } &\parbox[t]{1.95cm}{ } &\parbox[t]{1.95cm}{ 36 } &\\
\cellcolor[rgb]{1.000,0.871,0.678}\parbox[t]{5.2cm}{ Funde insgesamt } &\cellcolor[rgb]{1.000,0.871,0.678}\parbox[t]{1.95cm}{ } &\cellcolor[rgb]{1.000,0.871,0.678}\parbox[t]{1.95cm}{ 381 } &\cellcolor[rgb]{1.000,0.871,0.678}\parbox[t]{1.95cm}{ } &\cellcolor[rgb]{1.000,0.871,0.678}\parbox[t]{1.95cm}{ 141 } &\\
\multicolumn{5}{p{13.0cm}}{\parbox[t]{13.0cm}{ {\small (Stand der Tabelle: Sun, 26 Jun 2011 08:40:24 +0000)} }} &\\
\hline
\end{longtable}
\end{table}
 % Encountered a h3 tag
\subsection{% Ignoring editsection span
 % Encountered a span tag
%   Attributes: {u'class': u'mw-headline', u'id': u'Neu_hinzugekommene_Plagiatsquellen'}
 Neu hinzugekommene Plagiatsquellen }
 % Encountered a p tag

Die Quellenanalyse förderte eine Vielzahl von Textübernahmen aus vorher nicht untersuchten Quellen zu Tage. Die Plagiatsfunde aus diesen Quellen verteilen sich wie folgt: 

 % Encountered a table tag
%   Attributes: {u'class': u'wikitable'}
% Encountered a caption tag
% Encountered a br tag
% Encountered a tr tag
% Encountered a th tag
%   Attributes: {u'style': u'background-color:#ffdead; width:50%;'}
% Encountered a th tag
%   Attributes: {u'style': u'background-color:#ffdead;'}
% Encountered a br tag
% Encountered a th tag
%   Attributes: {u'style': u'background-color:#ffdead;'}
% Encountered a br tag
% Encountered a tr tag
%   Attributes: {u'style': u'text-align:right'}
% Encountered a td tag
%   Attributes: {u'style': u'text-align:left'}
% Encountered a a tag
%   Attributes: {u'href': u'/wiki/Kategorie:Behrmann_1984', u'title': u'Kategorie:Behrmann 1984'}
% Encountered a td tag
% Encountered a td tag
% Encountered a tr tag
%   Attributes: {u'style': u'text-align:right'}
% Encountered a td tag
%   Attributes: {u'style': u'text-align:left'}
% Encountered a a tag
%   Attributes: {u'href': u'/wiki/Kategorie:Bracher_1964', u'title': u'Kategorie:Bracher 1964'}
% Encountered a td tag
% Encountered a td tag
% Encountered a tr tag
%   Attributes: {u'style': u'text-align:right'}
% Encountered a td tag
%   Attributes: {u'style': u'text-align:left'}
% Encountered a a tag
%   Attributes: {u'href': u'/wiki/Kategorie:Bracher_1978', u'title': u'Kategorie:Bracher 1978'}
% Encountered a td tag
% Encountered a td tag
% Encountered a tr tag
%   Attributes: {u'style': u'text-align:right'}
% Encountered a td tag
%   Attributes: {u'style': u'text-align:left'}
% Encountered a a tag
%   Attributes: {u'href': u'/wiki/Kategorie:Bracher_1981', u'title': u'Kategorie:Bracher 1981'}
% Encountered a td tag
% Encountered a td tag
% Encountered a tr tag
%   Attributes: {u'style': u'text-align:right'}
% Encountered a td tag
%   Attributes: {u'style': u'text-align:left'}
% Encountered a a tag
%   Attributes: {u'href': u'/wiki/Kategorie:Commager_1952', u'title': u'Kategorie:Commager 1952'}
% Encountered a td tag
% Encountered a td tag
% Encountered a tr tag
%   Attributes: {u'style': u'text-align:right'}
% Encountered a td tag
%   Attributes: {u'style': u'text-align:left'}
% Encountered a a tag
%   Attributes: {u'href': u'/wiki/Kategorie:Craig_1984', u'title': u'Kategorie:Craig 1984'}
% Encountered a td tag
% Encountered a td tag
% Encountered a tr tag
%   Attributes: {u'style': u'text-align:right'}
% Encountered a td tag
%   Attributes: {u'style': u'text-align:left'}
% Encountered a a tag
%   Attributes: {u'href': u'/wiki/Kategorie:Fabian_1957', u'title': u'Kategorie:Fabian 1957'}
% Encountered a td tag
% Encountered a td tag
% Encountered a tr tag
%   Attributes: {u'style': u'text-align:right'}
% Encountered a td tag
%   Attributes: {u'style': u'text-align:left'}
% Encountered a a tag
%   Attributes: {u'href': u'/wiki/Kategorie:Fraenkel_1959', u'title': u'Kategorie:Fraenkel 1959'}
% Encountered a td tag
% Encountered a td tag
% Encountered a tr tag
%   Attributes: {u'style': u'text-align:right'}
% Encountered a td tag
%   Attributes: {u'style': u'text-align:left'}
% Encountered a a tag
%   Attributes: {u'href': u'/wiki/Kategorie:Fraenkel_1981', u'title': u'Kategorie:Fraenkel 1981'}
% Encountered a td tag
% Encountered a td tag
% Encountered a tr tag
%   Attributes: {u'style': u'text-align:right'}
% Encountered a td tag
%   Attributes: {u'style': u'text-align:left'}
% Encountered a a tag
%   Attributes: {u'href': u'/wiki/Kategorie:M%C3%BCller_1950', u'title': u'Kategorie:M\xfcller 1950'}
% Encountered a td tag
% Encountered a td tag
% Encountered a tr tag
%   Attributes: {u'style': u'text-align:right'}
% Encountered a td tag
%   Attributes: {u'style': u'text-align:left'}
% Encountered a a tag
%   Attributes: {u'href': u'/wiki/Kategorie:Pfl%C3%BCger_1983', u'title': u'Kategorie:Pfl\xfcger 1983'}
% Encountered a td tag
% Encountered a td tag
% Encountered a tr tag
%   Attributes: {u'style': u'text-align:right'}
% Encountered a td tag
%   Attributes: {u'style': u'text-align:left'}
% Encountered a a tag
%   Attributes: {u'href': u'/wiki/Kategorie:Seattle_1984', u'title': u'Kategorie:Seattle 1984'}
% Encountered a td tag
% Encountered a td tag
% Encountered a tr tag
% Encountered a td tag
%   Attributes: {u'colspan': u'3', u'style': u'text-align:center'}
% Encountered a small tag
\begin{table}[htp]
\centering
\caption{ Fundstellen aus\newline
neu entdeckten Quellen }
\begin{longtable}{|p{6.5cm}|p{3.25cm}|p{3.25cm}|p{0cm}}
\hline
\cellcolor[rgb]{1.000,0.871,0.678}\parbox[t]{6.5cm}{\textbf{ Quelle }} &\cellcolor[rgb]{1.000,0.871,0.678}\parbox[t]{3.25cm}{\textbf{ Anzahl\newline
Fragmente }} &\cellcolor[rgb]{1.000,0.871,0.678}\parbox[t]{3.25cm}{\textbf{ auf Anzahl\newline
Seiten }} &\\
\parbox[t]{6.5cm}{ \href{http://de.vroniplag.wikia.com/wiki/Kategorie:Behrmann_1984}{Behrmann 1984} } &\parbox[t]{3.25cm}{ 1 } &\parbox[t]{3.25cm}{ 1 } &\\
\parbox[t]{6.5cm}{ \href{http://de.vroniplag.wikia.com/wiki/Kategorie:Bracher_1964}{Bracher 1964} } &\parbox[t]{3.25cm}{ 11 } &\parbox[t]{3.25cm}{ 6 } &\\
\parbox[t]{6.5cm}{ \href{http://de.vroniplag.wikia.com/wiki/Kategorie:Bracher_1978}{Bracher 1978} } &\parbox[t]{3.25cm}{ 1 } &\parbox[t]{3.25cm}{ 1 } &\\
\parbox[t]{6.5cm}{ \href{http://de.vroniplag.wikia.com/wiki/Kategorie:Bracher_1981}{Bracher 1981} } &\parbox[t]{3.25cm}{ 1 } &\parbox[t]{3.25cm}{ 1 } &\\
\parbox[t]{6.5cm}{ \href{http://de.vroniplag.wikia.com/wiki/Kategorie:Commager_1952}{Commager 1952} } &\parbox[t]{3.25cm}{ 29 } &\parbox[t]{3.25cm}{ 13 } &\\
\parbox[t]{6.5cm}{ \href{http://de.vroniplag.wikia.com/wiki/Kategorie:Craig_1984}{Craig 1984} } &\parbox[t]{3.25cm}{ 5 } &\parbox[t]{3.25cm}{ 2 } &\\
\parbox[t]{6.5cm}{ \href{http://de.vroniplag.wikia.com/wiki/Kategorie:Fabian_1957}{Fabian 1957} } &\parbox[t]{3.25cm}{ 10 } &\parbox[t]{3.25cm}{ 7 } &\\
\parbox[t]{6.5cm}{ \href{http://de.vroniplag.wikia.com/wiki/Kategorie:Fraenkel_1959}{Fraenkel 1959} } &\parbox[t]{3.25cm}{ 5 } &\parbox[t]{3.25cm}{ 2 } &\\
\parbox[t]{6.5cm}{ \href{http://de.vroniplag.wikia.com/wiki/Kategorie:Fraenkel_1981}{Fraenkel 1981} } &\parbox[t]{3.25cm}{ 1 } &\parbox[t]{3.25cm}{ 1 } &\\
\parbox[t]{6.5cm}{ \href{http://de.vroniplag.wikia.com/wiki/Kategorie:M\%C3\%BCller_1950}{Müller 1950} } &\parbox[t]{3.25cm}{ 12 } &\parbox[t]{3.25cm}{ 5 } &\\
\parbox[t]{6.5cm}{ \href{http://de.vroniplag.wikia.com/wiki/Kategorie:Pfl\%C3\%BCger_1983}{Pflüger 1983} } &\parbox[t]{3.25cm}{ 19 } &\parbox[t]{3.25cm}{ 11 } &\\
\parbox[t]{6.5cm}{ \href{http://de.vroniplag.wikia.com/wiki/Kategorie:Seattle_1984}{Seattle 1984} } &\parbox[t]{3.25cm}{ 1 } &\parbox[t]{3.25cm}{ 1 } &\\
\multicolumn{3}{p{13.0cm}}{\parbox[t]{13.0cm}{ {\small (Stand der Tabelle: Sun, 26 Jun 2011 08:40:24 +0000)} }} &\\
\hline
\end{longtable}
\end{table}
 % Encountered a p tag

Aus den neu entdeckten Plagiatsquellen sticht --- auch durch den Umfang der Übernahme --- die Doktorarbeit des damaligen Ehemanns von Margarita Mathiopoulos, % Encountered a a tag
%   Attributes: {u'href': u'/wiki/Kategorie:Pfl%C3%BCger_1983', u'title': u'Kategorie:Pfl\xfcger 1983'}
\href{http://de.vroniplag.wikia.com/wiki/Kategorie:Pfl\%C3\%BCger_1983}{Friedbert Pflüger}, heraus. Pflüger war vier Jahre zuvor ebenfalls bei K.D. Bracher an der Universität Bonn über ein Amerika-Thema promoviert worden. Ein elf Seiten der Dissertation umfassendes Unterkapitel (% Encountered a a tag
%   Attributes: {u'href': u'/wiki/Mm/Mathiopoulos-1987/278', u'title': u'Mm/Mathiopoulos-1987/278'}
\href{http://de.vroniplag.wikia.com/wiki/Mm/Mathiopoulos-1987/278}{Seiten 278-288}) wird nahezu vollständig aus zehn Seiten von Pflügers Arbeit (dort Seiten 32-34 und 36-42) sowie Abschnitten aus einem Aufsatz von % Encountered a a tag
%   Attributes: {u'href': u'/wiki/Kategorie:Craig_1984', u'title': u'Kategorie:Craig 1984'}
\href{http://de.vroniplag.wikia.com/wiki/Kategorie:Craig_1984}{Gordon A. Craig} collagiert, dem Verfasser des Vorworts zur amerikanischen Ausgabe der Dissertation. 

% Encountered a p tag

Auch aus drei Werken des Doktorvaters Karl Dietrich Bracher wird unbelegt zitiert oder nicht sachgerecht entnommen. 

 % Encountered a h2 tag
\section{% Ignoring editsection span
 % Encountered a span tag
%   Attributes: {u'class': u'mw-headline', u'id': u'Beobachtungen_zum_Umgang_mit_Quellen'}
Beobachtungen zum Umgang mit Quellen}
 % Encountered a h3 tag
\subsection{% Ignoring editsection span
 % Encountered a span tag
%   Attributes: {u'class': u'mw-headline', u'id': u'Verschleierung_von_Text_und_Quellenangaben'}
Verschleierung von Text und Quellenangaben}
 % Encountered a p tag

Die hier als „Bauernopfer“ klassifizierten Textübernahmen entsprechen überwiegend nicht dem klassischen Muster eines Bauernopfers, bei dem wenigstens ein Satz der Quelle als Zitat gekennzeichnet wird, die eigentliche Übernahme aber weitaus umfangreicher ist. 

% Encountered a p tag

Vorliegend werden systematisch mehr oder weniger lange Abschnitte (einen Satz bis zu mehreren Seiten) einer Quelle leicht verändert („verschleiert“) und ohne Anführungszeichen in den Textkörper übernommen. Innerhalb oder am Ende der adaptierten Übernahme wird dann eine Fußnote eingefügt, die auf die Quelle (evtl. sogar mit Seitenzahl) hinweist, ohne den Umfang und Ursprung der Übernahme jedoch hinreichend genau zu kennzeichnen. Neben dem eigentlichen Text wird so auch die Referenz auf die Quelle verschleiert.% Encountered a sup tag
%   Attributes: {u'id': u'cite_ref-0', u'class': u'reference'}
% Encountered a a tag
%   Attributes: {u'href': u'#cite_note-0'}
\footnote{ Der Plagiatsgutachter Stefan Weber bezeichnet diese Art der Textübernahme als „Kennzeichnungsplagiat“ (vgl. Gutachten zur Dissertation Hahn: \url{http://www.gruene.at/uploads/media/GutachtenWeber.pdf})} 

% Encountered a p tag

Verschärft wird die Problematik durch die Anführung nicht nur einer Quelle, sondern einer Vielzahl von Quellen als mögliche Referenz für einen adaptierten Text, jeweils mit einem Hinweis wie „vgl. ..., vgl. ..., vgl, ...“. 

% Encountered a p tag

Beispiele: 

 % Encountered a ul tag
\begin{itemize}
% Encountered a li tag
\item Gleich die erste Fußnote in Teil I der Arbeit offenbart dieses problematische Verhältnis zur Zitiertechnik. Einem Gedankengang des Doktorvaters werden pauschal drei seiner Werke zugeordnet. 
\end{itemize}
 % Encountered a dl tag
\begin{description}
% Encountered a dd tag
\item % Encountered a dl tag
\begin{description}
% Encountered a dd tag
\item  % Encountered a a tag
%   Attributes: {u'href': u'http://de.vroniplag.wikia.com/wiki/Mm/Fragment_019_03-08', u'class': u'free'}
\url{http://de.vroniplag.wikia.com/wiki/Mm/Fragment_019_03-08} \end{description}
 \end{description}
 % Encountered a ul tag
\begin{itemize}
% Encountered a li tag
\item Besonders frappierend wirkt dieser Umgang mit einer Quelle, wenn an zwei unterschiedlichen Stellen auf einen identischen Grundtext zurückgegriffen wird. Verwendung von Krakau, S. 37, 3-16 und Fußnoten: 
\end{itemize}
 % Encountered a dl tag
\begin{description}
% Encountered a dd tag
\item % Encountered a dl tag
\begin{description}
% Encountered a dd tag
\item  % Encountered a a tag
%   Attributes: {u'href': u'http://de.vroniplag.wikia.com/wiki/Mm/Fragment_175_23-36', u'class': u'free'}
\url{http://de.vroniplag.wikia.com/wiki/Mm/Fragment_175_23-36} % Encountered a dd tag
 % Encountered a a tag
%   Attributes: {u'href': u'http://de.vroniplag.wikia.com/wiki/Mm/Fragment_085_02-16', u'class': u'free'}
\url{http://de.vroniplag.wikia.com/wiki/Mm/Fragment_085_02-16} \end{description}
 \end{description}
 % Encountered a p tag

Der Leser ist so nicht in der Lage zu erkennen, ob und in welchem Maße ein formulierter Gedankengang von einem anderen Autoren übernommen wurde, und muss von einer Eigenleistung des Übernehmenden ausgehen. Erst bei genauer Beschäftigung mit den angegebenen Quellen geht im Rahmen einer Textsynopse, wie sie zum Beispiel auf VroniPlag durchgeführt wird, Art und Umfang einer tatsächlichen Autorschaft hervor. 

% Encountered a p tag

Da diese Art der Zitierung systematisch geschieht, ist eine versehentlich falsche oder fahrlässig unvollständige Zitierung ausgeschlossen. Zu erklären ist eine solche Quellenverwendung entweder mit der Absicht, Autorschaft zu verschleiern, oder mit einem grundlegend mangelhaften Verständnis für Funktion und Sinn von Regeln wissenschaftlichen Arbeitens. Eine derart uneindeutige Quellenreferenzierung mag vielleicht für ein „mind map“ im Kontext eines zu bearbeitenden Konzeptes fruchtbar sein, für eine wissenschaftliche Veröffentlichung, insbesondere eine Dissertation, ist sie aber weder hinnehmbar noch geeignet. 

% Encountered a p tag

--------- 

 % Encountered a ol tag
%   Attributes: {u'class': u'references'}
% Encountered a li tag
%   Attributes: {u'id': u'cite_note-0'}
% Encountered a a tag
%   Attributes: {u'href': u'#cite_ref-0'}
% Encountered a a tag
%   Attributes: {u'href': u'http://www.gruene.at/uploads/media/GutachtenWeber.pdf', u'class': u'external free', u'rel': u'nofollow'}
 % Encountered a h3 tag
\subsection{% Ignoring editsection span
 % Encountered a span tag
%   Attributes: {u'class': u'mw-headline', u'id': u'W.C3.B6rtliche_Zitierung'}
 Wörtliche Zitierung }
 % Encountered a p tag

Eine Beliebigkeit in der Zitat-Auswahl und das Versäumnis ihrer methodischen Durchdringung wurde bereits von Shell in seiner Rezension in den Amerikastudien 1991 hervorgehoben. 

% Encountered a p tag

Diese Zitate nehmen insgesamt über 35 von 350 Seiten des Textteils in Anspruch. Dabei finden sich neben der Einbindung in den laufenden Text viele umfangreiche Zitate, die komplette Absätze umfassen und durch Einrückung und eine kleinere Schrifttype ausgezeichnet sind, jeweils mit abschließender Fußnote. Das umfangreichste Einzelzitat („Rede des Chief Seattle“, % Encountered a a tag
%   Attributes: {u'href': u'/wiki/Mm/Mathiopoulos-1987/245', u'title': u'Mm/Mathiopoulos-1987/245'}
\href{http://de.vroniplag.wikia.com/wiki/Mm/Mathiopoulos-1987/245}{S. 245f}) umfasst dabei 49 Zeilen. Zahlreiche weitere Zitate, insbesondere solche amerikanischer Präsidenten, haben einen Umfang von jeweils 10-20 Textzeilen. 

% Encountered a p tag

Die Auswahl und Reihenfolge der Zitate entspricht vielfach derjenigen in der gerade benutzten Quelle. Neben der ausufernden Erweiterung erfolgt bei solchen Zitierungen häufig der Rückgriff auf die englischen Originaltexte. Dieses Verfahren wird jedoch nicht immer durchgehalten, was ein Indiz für ein Plagiat ist. 

% Encountered a p tag

Beispiel: 

 % Encountered a ul tag
\begin{itemize}
% Encountered a li tag
\item Aus dem Original mit übernommene Zitate werden nur teilweise ins Englische übertragen 
\end{itemize}
 % Encountered a dl tag
\begin{description}
% Encountered a dd tag
\item % Encountered a dl tag
\begin{description}
% Encountered a dd tag
\item  % Encountered a a tag
%   Attributes: {u'href': u'http://de.vroniplag.wikia.com/wiki/Mm/Fragment_287_20-46', u'class': u'free'}
\url{http://de.vroniplag.wikia.com/wiki/Mm/Fragment_287_20-46} \end{description}
 \end{description}
 % Encountered a h3 tag
\subsection{% Ignoring editsection span
 % Encountered a span tag
%   Attributes: {u'class': u'mw-headline', u'id': u'Stilwechsel'}
Stilwechsel}
 % Encountered a p tag

An den eingerückten wörtlichen Zitaten lässt sich ein Stilwechsel innerhalb der Arbeit festmachen. Während --- abgesehen von einem eingerückten Zitat in der Einleitung --- der Teil I der Arbeit (55 Seiten) noch mit gerade mal 3 eingerückten Zitaten mit einer Gesamtlänge von 15 Zeilen auskommt, entfallen auf die Teile II bis IV (218 Seiten) über 300 eingerückte Zitate mit ca. 1700 Zeilen Umfang. Bemerkenswert ist dabei, dass sich die Plagiatsfunde bisher ganz überwiegend ebenfalls auf die Teile II-IV der Arbeit beschränken, in denen massiv eingerückt zitiert wird. 

% Encountered a p tag

Dieser Stilwechsel im Text ist mit einer Änderung der Arbeitstechnik bei der Abfassung der Arbeit verbunden: ab Teil II der Arbeit wird großflächig wörtlich zitiert und in großen Abschnitten plagiiert. 

 % Encountered a h2 tag
\section{% Ignoring editsection span
 % Encountered a span tag
%   Attributes: {u'class': u'mw-headline', u'id': u'Ausblick'}
 Ausblick }
 % Encountered a p tag

Die Untersuchung der Dissertation ist noch nicht abgeschlossen. 

% Encountered a p tag

Während der Textabgleich mit der verwendeten deutschsprachigen Literatur weit fortgeschritten ist, ist die Analyse der fremdsprachigen Texte (Übersetzungsplagiat) noch am Anfang und kann gerade für den ersten Teil der Arbeit noch Ergebnisse bringen. 

% Encountered a p tag

Die Sichtung der Fundstellen durch Zweitsichter (Vier-Augen-Prinzip) schreitet voran. 

 % Encountered a h2 tag
\section{% Ignoring editsection span
 % Encountered a span tag
%   Attributes: {u'class': u'mw-headline', u'id': u'Anhang'}
Anhang}
 % Encountered a h3 tag
\subsection{% Ignoring editsection span
 % Encountered a span tag
%   Attributes: {u'class': u'mw-headline', u'id': u'Definition_von_Plagiatskategorien'}
 Definition von Plagiatskategorien }
 % Encountered a p tag

Für eine detailliertere Analyse der Fundstellen wird zwischen verschiedenen Arten von Plagiaten unterschieden. Diese Plagiatskategorien basieren auf den Ausarbeitungen von Weber-Wulff und Wohnsdorf: „Strategien und Plagiatsbekämpfung“ (Information --- Wissenschaft \& Praxis (2006) 2, S. 90-98). Eine vollständige Beschreibung der hier verwendeten Kategorien ist unter % Encountered a a tag
%   Attributes: {u'href': u'http://de.vroniplag.wikia.com/wiki/VroniPlag_Wiki:PlagiatsKategorien', u'class': u'free'}
\url{http://de.vroniplag.wikia.com/wiki/VroniPlag_Wiki:PlagiatsKategorien} aufgeführt. Die Plagiatskategorien sind im Einzelnen: 

 % Encountered a h4 tag
\subsubsection{% Ignoring editsection span
 % Encountered a span tag
%   Attributes: {u'class': u'mw-headline', u'id': u'Komplettplagiat'}
 Komplettplagiat }
 % Encountered a p tag

Komplette Abschnitte der Quelle wurden wörtlich und ohne Zitat übernommen. Dabei wird unterschieden, ob die Quelle an einer anderen Stelle im Literaturverzeichnis auftaucht (und damit die Fußnote und Anführungszeichen nur evtl. „vergessen“ wurden) oder ob die Quelle überhaupt nicht erwähnt ist. 

 % Encountered a h4 tag
\subsubsection{% Ignoring editsection span
 % Encountered a span tag
%   Attributes: {u'class': u'mw-headline', u'id': u'Verschleierung'}
 Verschleierung }
 % Encountered a p tag

Verschleierungen sind Textstellen, die erkennbar von fremden Quellen abstammen, aber umformuliert und weder als Paraphrase noch als Zitat kenntlich gemacht wurden. Gemeint sind Texte, die wegen der Umformulierung nicht mehr einfach als „Gänsefüßchen/Fußnote vergessen“ abgetan werden können. Es liegt die Vermutung nahe, dass die Neuformulierung dazu dient, die Herkunft von Text und Gedanken aus fremder Quelle zu verschleiern (d.h. z.B. Suche zu erschweren, Stil und Duktus anzupassen, damit es zur restlichen Arbeit passt, etc.). 

 % Encountered a h4 tag
\subsubsection{% Ignoring editsection span
 % Encountered a span tag
%   Attributes: {u'class': u'mw-headline', u'id': u'Bauernopfer'}
 Bauernopfer }
 % Encountered a p tag

In diesem Fall wird zwar eine Fußnote angegeben, diese bezieht sich jedoch auf einen unbedeutenden Teil eines Originaltextes, während größere Abschnitte aus demselben ohne Zitatnachweis übernommen werden und damit den Eindruck einer eigenen Denkleistung erwecken. 

% Encountered a p tag

Bei einem „Verschärften Bauernopfer“ wird zusätzlich die exakte Herkunft aus der exakten Quelle durch Zusätze in der Fußnote wie „vgl.“ oder „siehe auch“ weiter verschleiert. 




\appendix
\section{Textnachweise}



\renewcommand{\bibname}{Quellenverzeichnis}
\bibliographystyle{dinat-custom}
\bibliography{ab}
\end{document}

